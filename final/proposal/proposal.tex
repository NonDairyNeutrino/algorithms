\documentclass{article}
% =======PACKAGES=======
% FORMATTING
\usepackage[margin=0.625in]{geometry}
\usepackage{parskip, setspace}
\setstretch{1.15}
% TYPESETTING - MATH
\usepackage{amsmath, amsfonts}
\usepackage{amsthm}
\usepackage[ruled, linesnumbered, noend]{algorithm2e}
\NewCommandCopy{\legacyunderscore}{\_}
\renewcommand{\_}{\ifincsname_\else\legacyunderscore\fi}
\usepackage{listings}
\usepackage{xcolor}

\lstdefinestyle{mystyle}{
    backgroundcolor=\color{lightgray},   
    commentstyle=\color{darkgray},
    keywordstyle=\color{red},
    numberstyle=\color{black},
    stringstyle=\color{violet},
    basicstyle=\ttfamily\footnotesize,
    breakatwhitespace=false,         
    breaklines=true,                 
    captionpos=b,                    
    keepspaces=true,                 
    numbers=left,                    
    numbersep=5pt,                  
    showspaces=false,                
    showstringspaces=false,
    showtabs=false,                  
    tabsize=2
}
\lstset{style=mystyle}
% RICH
\usepackage{graphicx, caption}
\usepackage{hyperref}
% BIBLIOGRAPHY
\usepackage[
backend=biber,
sorting=ynt
]{biblatex}
\addbibresource{bib.bib}

\newcommand{\integer}{\textbf{int} }

% =======TITLE=======
\title{\vspace*{-0.625in}CS 529: Advanced Data Structures \& Algorithms \\ Final Project Proposal}
\author{Nathan Chapman, Hunter Lawrence, Andrew Struthers}
\date{\today}

\begin{document}

    \maketitle

\section*{Introduction}

\section*{Genetic Algorithm}
A Genetic Algorithm is a type of evolutionary algorithm that attempts to follow the biological process of evolution. This algorithm works by first having an initial population generated randomly or by some smarter heuristic. The initial population then goes through typical genetic processes such as tournament or elitism selection, genetic crossover, and random mutation. Each member of the population is represented by a ``genome sequence", which encodes the some solution to an optimization problem somehow. The genetic processes operate on these genome sequences to form children. The children form the new population, and the cycle repeats, until some stopping criteria is met. As was pointed out in class, a lot of the decisions involved in using a GA to solve a problem are pretty arbitrary and disconnected from biological genetics. With this project, we want to determine the reason behind the usually arbitrary decisions when choosing a selection algorithm, a crossover algorithm, a mutation algorithm, and the hyperparameters of the model. We will analyze typical algorithms used for these processes and make decisions based off of problem context and evolutionary context to remove the arbitrary nature of these choices. Additionally, we will use the Fast Fourier Transform to determine the stopping criteria of the evolution process, as described below. 
\section*{Fast Fourier Transform}
\section*{Parallelization}
WHAT PARTS OF THIS PROJECT CAN BE DONE IN PARALLEL, WHAT WE WOULD USE PARALLELIZATION FOR, AND WHY WE CARE ABOUT PARALLELIZING THIS SHIT
\section*{Bringing It All Together}

\end{document}