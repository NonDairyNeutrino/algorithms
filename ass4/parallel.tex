\documentclass{article}
% =======PACKAGES=======
% FORMATTING
\usepackage[margin=0.625in]{geometry}
\usepackage{parskip, setspace}
\setstretch{1.15}
% TYPESETTING - MATH
\usepackage{amsmath, amsfonts}
\usepackage[ruled, linesnumbered, noend]{algorithm2e}
\NewCommandCopy{\legacyunderscore}{\_}
\renewcommand{\_}{\ifincsname_\else\legacyunderscore\fi}
\usepackage{listings}
\usepackage{xcolor}

\lstdefinestyle{mystyle}{
    backgroundcolor=\color{lightgray},   
    commentstyle=\color{darkgray},
    keywordstyle=\color{red},
    numberstyle=\color{black},
    stringstyle=\color{violet},
    basicstyle=\ttfamily\footnotesize,
    breakatwhitespace=false,         
    breaklines=true,                 
    captionpos=b,                    
    keepspaces=true,                 
    numbers=left,                    
    numbersep=5pt,                  
    showspaces=false,                
    showstringspaces=false,
    showtabs=false,                  
    tabsize=2
}
\lstset{style=mystyle}
% RICH
\usepackage{graphicx, caption}
\usepackage{hyperref}
% BIBLIOGRAPHY
\usepackage[
backend=biber,
sorting=ynt
]{biblatex}
\addbibresource{bib.bib}

\newcommand{\integer}{\textbf{int} }

% =======TITLE=======
\title{\vspace*{-0.625in}CS 529: Advanced Data Structures \& Algorithms \\ Assignment 4: Parallel Algorithms}
\author{Nathan Chapman, Hunter Lawrence, Andrew Struthers}
\date{\today}

\begin{document}

    \maketitle

\begin{enumerate}
	\item Consider the problem of adding two $n\times n$ matrices. If it takes $t_a$ time for one person to add two numbers, how many people do we need to minimize the total time spent to get the final answer? What will be the minimum amount of time needed to find the answer, if we assume that we have enough people? Justify your answers.
	
        \underline{\textbf{Solution}}

        Given $n$ tasks that take equal time, the most efficient way to complete all of them would be for each task to be done at the same time.  Therefore, the number of processors needed to complete $n$ tasks is also $n$, as each processor executes each tasks simultaneously.

        The addition of two $n \times n$ matrices requires each of the $n^2$ elements to be added together, where each addition is a single task.  Therefore the addition of two $n \times n$ matrices is equivalent to a set of $n^2$ tasks.  By the above conclusion, the number of people required to execute the addition of two $n \times n$ matrices is $n^2$.

	\item Write a CREW PRAM algorithm for adding all n numbers in a list in $O(\log n)$ time.
	\begin{enumerate}
		\item break list into ``$\log n$'' chunks, each of length $n / \log n$.  Each summation takes $n / \log n$ time, but all happens at the same time.  Then exclusively write ``\verb|sum += subsum|'' $\log n$ times.  This is a CREW-PRAM summation in $\log n$ time.
	\end{enumerate}

	\item Write a PRAM algorithm using $n^3$ processors to multiply two $n\times n$ matrices. Your algorithm should run in $O(\log n)$ time.
	\begin{enumerate}
		\item
	\end{enumerate}

\end{enumerate}

\end{document}