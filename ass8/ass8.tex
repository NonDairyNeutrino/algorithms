\documentclass{article}
% =======PACKAGES=======
% FORMATTING
\usepackage[margin=0.625in]{geometry}
\usepackage{parskip, setspace}
\setstretch{1.15}
% TYPESETTING - MATH
\usepackage{amsmath, amsfonts, amsthm}
% TYPESETTING - ALGORITHMS
\usepackage[ruled, linesnumbered, noend]{algorithm2e}
\NewCommandCopy{\legacyunderscore}{\_}
\renewcommand{\_}{\ifincsname_\else\legacyunderscore\fi}
% TYPESETTING - CODE
\usepackage{listings}
\usepackage{xcolor}
\lstdefinestyle{mystyle}{
    backgroundcolor=\color{lightgray},   
    commentstyle=\color{darkgray},
    keywordstyle=\color{red},
    numberstyle=\color{black},
    stringstyle=\color{violet},
    basicstyle=\ttfamily\footnotesize,
    breakatwhitespace=false,         
    breaklines=true,                 
    captionpos=b,                    
    keepspaces=true,                 
    numbers=left,                    
    numbersep=5pt,                  
    showspaces=false,                
    showstringspaces=false,
    showtabs=false,                  
    tabsize=2
}
\lstset{style=mystyle}
% RICH
\usepackage{graphicx, caption}
\usepackage{hyperref}
% BIBLIOGRAPHY
% \usepackage[
% backend=biber,
% sorting=ynt
% ]{biblatex}
% \addbibresource{bib.bib}

\newcommand{\integer}{\textbf{int} } % for use in algorithm environments

% =======TITLE=======
\title{\vspace*{-0.625in}CS 529: Advanced Data Structures \& Algorithms \\ Assignment 8: Number Theory and NP Algorithms}
\author{Nathan Chapman, Hunter Lawrence, Andrew Struthers}
\date{\today}

\begin{document}

\maketitle

\section*{Number Theory}

    3 pages of summary of number theory and proof of some statement

    \subsection*{Euclidean Division}

        It can be shown that for any pair of positive integers $a, b \in \mathbb{Z}^+$, there exists positive integers $q, r \in \mathbb{Z}^+$ such that 

        \begin{equation}
            a = q b + r
        \end{equation}

        where $0 \leq r < b$.  $q$ is called the ``quotient'' and $r$ is called the ``remainder''.  Because this is true for any pair, the same holds for every pair of positive integers.  A well known algorithm for calculating the remaineder is long division.

    \subsection*{Modular Arithmetic}

    \subsection*{Modular Exponentiation}

    \subsection*{Modular Congruence}

    \subsection*{Modular Multiplicative Inverse}

    \subsection*{Carmichael's Totient}

    \subsection*{Extended Euclidean Algorithm}

    \subsection*{Primality Testing}

\section*{Theorem 11.6}

    Prove that the $\gcd(n, m)$ is a product of primes that are common to $n$ and $m$, where the power of each prime in the product is the smaller of its orders in $n$ and $m$.

    \begin{proof}
        
    \end{proof}

\section*{Presburger and Halting Complexity}

\section*{Polynomial-time Algorithms}

    List 3 and justify

\section*{Encoding Schemes}

    Give a problem and two encoding schemes for its input.  Express its performance using your encoding schemes.

\end{document}