\documentclass{article}
% =======PACKAGES=======
% FORMATTING
\usepackage[margin=0.625in]{geometry}
\usepackage{parskip, setspace}
\setstretch{1.15}
% TYPESETTING - MATH
\usepackage{amsmath, amsfonts}
\usepackage{amsthm}
\usepackage[ruled, linesnumbered, noend]{algorithm2e}
\NewCommandCopy{\legacyunderscore}{\_}
\renewcommand{\_}{\ifincsname_\else\legacyunderscore\fi}
\usepackage{listings}
\usepackage{xcolor}

\lstdefinestyle{mystyle}{
    backgroundcolor=\color{lightgray},   
    commentstyle=\color{darkgray},
    keywordstyle=\color{red},
    numberstyle=\color{black},
    stringstyle=\color{violet},
    basicstyle=\ttfamily\footnotesize,
    breakatwhitespace=false,         
    breaklines=true,                 
    captionpos=b,                    
    keepspaces=true,                 
    numbers=left,                    
    numbersep=5pt,                  
    showspaces=false,                
    showstringspaces=false,
    showtabs=false,                  
    tabsize=2
}
\lstset{style=mystyle}
% RICH
\usepackage{graphicx, caption}
\usepackage{hyperref}
% BIBLIOGRAPHY
\usepackage[
backend=biber,
sorting=ynt
]{biblatex}
\addbibresource{bib.bib}

\newcommand{\integer}{\textbf{int} }

% =======TITLE=======
\title{\vspace*{-0.625in}CS 529: Advanced Data Structures \& Algorithms \\ Assignment 5: The Fast Fourier Transform}
\author{Nathan Chapman, Hunter Lawrence, Andrew Struthers}
\date{\today}

\begin{document}

    \maketitle

    \section*{Multiplying Polynomials}

        Multiply the polynomials $A(x) = 7 x^3 - x^2 + x - 10$ and $B(x) = 8x^3 - 6x + 3$ using equations (30.1) and (30.2).

        \underline{\textbf{Solution}}

        
        


    \section*{$n$ degree polynomials need $n$ points}

        Prove that n distinct point-value pairs are necessary to uniquely specify a polynomial of degree-bound n, that is, if fewer than n distinct point-value pairs are given, they fail to specify a unique polynomial of degree-bound n. (Hint: Use Theorem 30.1)

\begin{proof}
To prove that $n$ distinct point-value pairs are necessary to uniquely specify a polynomial of degree-bound $n$, we can use the concept of polynomial interpolation.

Polynomial interpolation states that given $n+1$ distinct points $(x_0, y_0)$, $(x_1, y_1)$, $\hdots$, $(x_n, y_n)$, where $x_i$ are distinct, there exists a unique polynomial of degree at most $n$ that passes through all these points.

Let's suppose we have fewer than $n$ distinct point-value pairs, say $k$ pairs where $k < n$. Then we have $(k+1)$ distinct points $(x_0, y_0)$, $(x_1, y_1)$, $\hdots$, $(x_k, y_k)$. According to the interpolation theorem, there exists a unique polynomial of degree at most $k$ that passes through these points.

However, we want to specify a polynomial of degree-bound $n$. Since $k < n$, the polynomial we obtain through interpolation might have a degree less than $n$. This means it won't fit all the constraints we want to impose on our polynomial.

Therefore, if we have fewer than $n$ distinct point-value pairs, we cannot guarantee a polynomial of degree-bound $n$ uniquely, as there may be multiple polynomials that satisfy those points and have a degree less than $n$. Hence, we require at least $n$ distinct point-value pairs to uniquely specify a polynomial of degree-bound $n$.
\end{proof}

    \section*{Applications of the FFT}

\printbibliography

\end{document}
