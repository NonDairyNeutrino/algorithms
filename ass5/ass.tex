\documentclass{article}
% =======PACKAGES=======
% FORMATTING
\usepackage[margin=0.625in]{geometry}
\usepackage{parskip, setspace}
\setstretch{1.15}
% TYPESETTING - MATH
\usepackage{amsmath, amsfonts}
\usepackage{amsthm}
\usepackage[ruled, linesnumbered, noend]{algorithm2e}
\NewCommandCopy{\legacyunderscore}{\_}
\renewcommand{\_}{\ifincsname_\else\legacyunderscore\fi}
\usepackage{listings}
\usepackage{xcolor}

\lstdefinestyle{mystyle}{
    backgroundcolor=\color{lightgray},   
    commentstyle=\color{darkgray},
    keywordstyle=\color{red},
    numberstyle=\color{black},
    stringstyle=\color{violet},
    basicstyle=\ttfamily\footnotesize,
    breakatwhitespace=false,         
    breaklines=true,                 
    captionpos=b,                    
    keepspaces=true,                 
    numbers=left,                    
    numbersep=5pt,                  
    showspaces=false,                
    showstringspaces=false,
    showtabs=false,                  
    tabsize=2
}
\lstset{style=mystyle}
% RICH
\usepackage{graphicx, caption}
\usepackage{hyperref}
% BIBLIOGRAPHY
\usepackage[
backend=biber,
sorting=ynt
]{biblatex}
\addbibresource{bib.bib}

\newcommand{\integer}{\textbf{int} }

% =======TITLE=======
\title{\vspace*{-0.625in}CS 529: Advanced Data Structures \& Algorithms \\ Assignment 5: The Fast Fourier Transform}
\author{Nathan Chapman, Hunter Lawrence, Andrew Struthers}
\date{\today}

\begin{document}

    \maketitle

    \section*{Multiplying polynomials}

        Multiply the polynomials $A(x) = 7 x^3 - x^2 + x - 10$ and $B(x) = 8x^3 - 6x + 3$ using equations (30.1) and (30.2).

        \underline{\textbf{Solution}}

        Let there be polynomials $A$ and $B$ of degree $n = 3$ defined by 

        \begin{subequations}
            \begin{equation}
                A(x) = 7 x^3 - x^2 + x - 10 \implies A = \langle -10, 1, -1, 7 \rangle
            \end{equation}
            \begin{equation}
                B(x) = 8x^3 - 6x + 3 \implies B = \langle 3, -6, 0, 8\rangle
            \end{equation}
        \end{subequations}

        where the $\langle \cdot, \cdot \rangle$ is the coefficient representation of the polynomials in the vector space of polynomials.

        The polynomial $C$ such that $C(x) = A(x)B(x)$, can be constructed by applying the following relations from Cormen et. al. 

        \begin{subequations}
            \begin{equation}
                C(x) = \sum_{j = 0}^{2n - 2} c_j x^j
            \end{equation}
            \begin{equation}
                c_j = \sum_{k = 0}^{j} a_k b_{j - k}
            \end{equation}
        \end{subequations}

        Therefore,

        \begin{subequations}
            \begin{equation}
                c_0 = (-10)(3) = -30
            \end{equation}
            \begin{equation}
                c_1 = (-10)(-6) + (1)(3) = 63
            \end{equation}
            \begin{equation}
                c_2 = (-10)(0) + (1)(-6) + (-1)(3) = -9
            \end{equation}
            \begin{equation}
                c_3 = (-10)(8) + (1)(0) + (-1)(-6) + (7)(3) = -53
            \end{equation}
            \begin{equation}
                c_4 = 
            \end{equation}
            \begin{equation}
                c_5 =
            \end{equation}
            \begin{equation}
                c_6 = 
            \end{equation}
        \end{subequations}

        Thus the polynomial $C$ such that $C(x) = A(x)B(x)$ is defined by 

        \begin{equation}
            \boxed{C(x) = 56 x^6-8 x^5-34 x^4-53 x^3-9 x^2+63 x-30}
        \end{equation}

    \section*{$n$ degree polynomials need $n$ points}

        Prove that n distinct point-value pairs are necessary to uniquely specify a polynomial of degree-bound n, that is, if fewer than n distinct point-value pairs are given, they fail to specify a unique polynomial of degree-bound n. (Hint: Use Theorem 30.1)

\begin{proof}
We will show that we need $n$ distinct points to specify a unique polynomial of order $n$. In terms of polynomial interpolation, Theorem 30.1 from the Cormen textbook states that for any set $\left\{ (x_0, y_0), (x_1, y_1), \hdots,(x_{n-1}, y_{n-1})\right\}$, of $n$ points where $x_k$ are distinct, there exists a unique polynomial $P(x)$ of degree at most $n$ such that $y_k=P(x_k) \forall k=0, 1, \hdots, n-1$. We will prove that we need $n$ distinct points to specify a unique polynomial of order $n$.

Suppose we have fewer than $n$ distinct point-value pairs, say $k$ pairs where $k < n$. Then we have $(k+1)$ distinct points $\left\{(x_0, y_0), (x_1, y_1), \hdots, (x_k, y_k)\right\}$. According to Theorem 30.1, there exists a unique polynomial of degree at most $k$ that passes through these points.

However, we want to specify a polynomial of degree-bound $n$. Since $k < n$, the polynomial we obtain through interpolation will have a degree less than $n$. Thus, our $k$ distinct points resulted in a polynomial of degree $k$ which does not fit in our constraints of desiring an $n$ degree polynomial from $k$ points. 

Therefore, if we have fewer than $n$ distinct point-value pairs, by Theorem 30.1 we cannot guarantee a polynomial of degree-bound $n$ uniquely, as there may be multiple polynomials that satisfy those points and have a degree less than $n$. Hence, we require at least $n$ distinct point-value pairs to uniquely specify a polynomial of degree-bound $n$.
\end{proof}

    \section*{Applications of the FFT}

        \begin{itemize}
            \item A function can be represented as a power series, and approximated with a truncated power series.  The Taylor series is a special case of this.
            \item A function that is the product of other functions can thus be represented as a product of the truncated power series/polynomials.
            \item Additionally, there are other similar representations as a series of orthogonal polynomials like the Hermite polynomials.
            \item Such representations and approximations are ubiquitous in
            \item (Wikipedia: Hermite Polynomials) The polynomials arise in:
                \begin{itemize}
                    \item signal processing as Hermitian wavelets for wavelet transform analysis
                    \item probability, such as the Edgeworth series, as well as in connection with Brownian motion;
                    \item combinatorics, as an example of an Appell sequence, obeying the umbral calculus;
                    \item numerical analysis as Gaussian quadrature;
                    \item physics, where they give rise to the eigenstates of the quantum harmonic oscillator; and they also occur in some cases of the heat equation;
                    \item systems theory in connection with nonlinear operations on Gaussian noise.
                    \item random matrix theory in Gaussian ensembles.
                \end{itemize}
        \end{itemize}

\printbibliography

\end{document}
