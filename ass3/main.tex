\documentclass{article}
% =======PACKAGES=======
% FORMATTING
\usepackage[margin=0.625in]{geometry}
\usepackage{parskip, setspace}
\setstretch{1.15}
% TYPESETTING - MATH
\usepackage{amsmath, amsfonts}
\usepackage[ruled, linesnumbered, noend]{algorithm2e}
\NewCommandCopy{\legacyunderscore}{\_}
\renewcommand{\_}{\ifincsname_\else\legacyunderscore\fi}
\usepackage{listings}
\usepackage{xcolor}

\lstdefinestyle{mystyle}{
    backgroundcolor=\color{lightgray},   
    commentstyle=\color{darkgray},
    keywordstyle=\color{red},
    numberstyle=\color{black},
    stringstyle=\color{violet},
    basicstyle=\ttfamily\footnotesize,
    breakatwhitespace=false,         
    breaklines=true,                 
    captionpos=b,                    
    keepspaces=true,                 
    numbers=left,                    
    numbersep=5pt,                  
    showspaces=false,                
    showstringspaces=false,
    showtabs=false,                  
    tabsize=2
}
\lstset{style=mystyle}
% RICH
\usepackage{graphicx, caption}
\usepackage{hyperref}
% BIBLIOGRAPHY
\usepackage[
backend=biber,
sorting=ynt
]{biblatex}
\addbibresource{bib.bib}

\newcommand{\integer}{\textbf{int} }

% =======TITLE=======
\title{\vspace*{-0.625in}CS 529: Advanced Data Structures \& Algorithms \\ Assignment 3: String Matching Algorithms \& Dynamic Programming}
\author{Nathan Chapman, Hunter Lawrence, Andrew Struthers}
\date{\today}

\begin{document}

    \maketitle

    \section*{Introduction}

        Let $s_1, s_2$ be strings of length $m, n$ respectively, and $(i, j) \in \mathbb{Z}_m \times \mathbb{Z}_n$ i.e. $i, j$ are integers such that $0 \leq i \leq m - 1, 0 \leq j \leq n - 1$.  We consider the \texttt{opt} algorithm as defined in algorithm \ref{alg:opt}.

        \begin{function}
            \caption{opt(\integer $i$, \integer $j$)}
            \label{alg:opt}
            \KwIn{Indices of strings}
            \KwOut{Interger cost of best string alignment}
            
            \If{i = m}{
                \Return{2(n - j)}
            }
            \eIf{j = n}{
                \Return{2(m - i)}
            }{
                \eIf{$s_1$[i] = $s_2$[j]}{
                    $penalty \gets 0$
                }{
                    $penalty \gets 1$
                }
                \Return{$\min$(\opt(i + 1, j + 1) + penalty, \opt(i + 1, j) + 2, \opt(i, j + 1) + 2)}
            }
        \end{function}

    \section*{Runtime Complexity of \texttt{opt}}

    \texttt{opt(i, j)} seeks to find the penalty number for the optimal alignment of strings $s_1, s_2$ by exploring all possible penalty values which can be incurred by single character shifts. The results of this will allow a dynamic programming algorithm (specified later) to find out how to match the sequences according to this value.

    because \texttt{opt()}, as depicted above, incurs three separate recursive calls, only ending when either $i \leq  n $ OR $ j \leq m$. This results in an exponential worst case time efficiency of $O(3^n)$. 
\pagebreak
    \section*{Sequence Alignment with Dynamic Programming}

        While the \ref{alg:opt} algorithm directly has exponential complexity, it can be simply changed to use dynamic programming techniques, such as memoization i.e. storing results so they don't need to be calculated again, to greatly reduce the runtime.[DYNAMIC PROGRAMMING APPROACH IN WORDS HERE]

        Let $s_1, s_2$ be strings of length $m, n$ respectively, and $(i, j) \in \mathbb{Z}_m \times \mathbb{Z}_n$ i.e. $i, j$ are integers such that $0 \leq i \leq m - 1, 0 \leq j \leq n - 1$.  We consider the \texttt{opt\_dynamic} algorithm as defined in algorithm \ref{alg:opt_Dynamic}.

        \begin{function}
            \caption{opt\_dynamic(\integer $i$, \integer $j$)}
            \label{alg:opt_Dynamic}
            \KwIn{Indices of strings}
            \KwOut{Interger cost of best string alignment}
            $memo \gets empty\_dictionary$ \tcp{initialize tuple-cost dictionary}
            \If{(i, j) $\in$ memo}{
                \Return{$memo[(i, j)]$}
            }
            \If{i = m}{
                \Return{$2(n - j)$}
            }
            \eIf{j = n}{
                \Return{$2(m - i)$}
            }{
                \eIf{$s_1[i]$ = $s_2[j]$}{
                    $penalty \gets 0$
                }{
                    $penalty \gets 1$
                }
                $r \gets \min$(\opt$(i + 1, j + 1) + penalty$, \opt$(i + 1, j) + 2$, \opt$(i, j + 1) + 2)$\;
                $memo[(i, j)] \gets r$\;
                \Return{r}
            }
        \end{function}

    \section*{Finding an Optimal Alignment}

    \printbibliography

\end{document}